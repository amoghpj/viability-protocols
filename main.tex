\documentclass{article}

\usepackage{graphicx}
\usepackage{amsmath}
\usepackage{soul}
\usepackage{cleveref}
\usepackage{xcolor}
\usepackage[]{fancyhdr}
\title{Protocol and tasks - MUPPETS viability}
\newcounter{comments}
\newcommand{\amogh}[1]{{\addtocounter{comments}{1}}{\color{red}{[\textbf{TODO Amogh \thecomments :}\ #1}]}}
\newcommand{\bradley}[1]{{\addtocounter{comments}{1}}{\color{violet}{[\textbf{TODO Bradley \thecomments :}\ #1]}}}
\newcommand{\zeina}[1]{{\addtocounter{comments}{1}}{\color{blue}{[\textbf{TODO Zeina \thecomments :}\ #1]}}}
\newcommand{\colin}[1]{{\addtocounter{comments}{1}}{\color{cyan}{[\textbf{TODO Colin \thecomments :}\ #1]}}}
\date{\today}


\begin{document}
\pagestyle{fancy}
\fancyfoot{}
\fancyhead{}
\fancyfoot[LO,RE]{\thepage\\LAST UPDATED \today}

\maketitle
\tableofcontents


\section{Task List}

\amogh{Make high throughput bacterial/fungal serial dilution script}

\amogh{Make high throughput viral serial dilution script}

\amogh{Make flexible script to reorganize samples after sample identification?}

\zeina{Be Bad Cop Again}

\bradley{Flesh out the day-by-day courses with Amogh}

\colin{Add in the protocol for difficult fungal handling}

\amogh{Fill out protocols}

\newpage

\section{Foreword}
We are receiving 60 distinct samples, with 2 replicates from NIST. These samples are 10 mL blood which are potentially frozen, and potentially not frozen. Each sample has between 0 and 3 organisms, which are a combination of fungi, viruses, and bacteria. We need to store the sample in 2 CONOPS (doubling treatment plates). The Cell Viability Core (CVC) are potentially unblinded to the samples. The treatment teams (Silver \& Marelli) are blinded to the samples to avoid preferential treatment. We need to report for each sample what was in it, how much, and how much survived after 7 days.

All samples need to be treated as BSL2+. All well plates need to have lids at all times. All handling of samples need to be done in the Tecan (sealed) or in a bio-safety cabinet. All biosafety bags which handle BSL2+ samples need to be sprayed down before placing in normal biosafety containers, including that from the Tecan. \amogh{I was forgetting some of the common sense safety stuff you probably know when converting from BSC best practices to Tecan best practices, perhaps it would be good to write them down}

\section{Protocols}
\begin{itemize}
    \item Media prep (\Cref{sec:media-prep})
    \item Cell culture calculations
    \item Tecan startup (\Cref{sec:tecan-start})
    \item Tecan shutdown (\Cref{sec:tecan-shut})
    \item Protocol - Bacterial, fungal serial dilutions (Method: \texttt{Serial Dilutions - Bacterial/Fungal})
    \item Protocol - Viral serial dilutions
    \item Protocol - qPCR setup for bacteria/fungi (Method: \texttt{384 well plates read})
    \item Protocol - qPCR setup for virus (Method: \texttt{384 well plates read})
    \item Protocol - OD reads (Method: \texttt{384 well plates read})
    \item Treatment Preparation and Conditions \bradley{}
    \item Handling of samples with hyphal fungi \colin{} 
\end{itemize}

\subsection{Media prep}
\label{sec:media-prep}
\begin{description}
\item[Universal bacterial media (UBM)] TSB+BHI supplemented with NAD and Hemin. TSB - Mix 30g TSB powder into 1L media, autoclave. BHI - mix 37g BHI powder into 1L media, autoclave. 2000X Hemin and NAD stocks are stored in the -80 in GOLD513.
\item[Fungal media] Potato Dextrose Broth.
\item[Cell culture media for Vero cells] EMEM + 10\% FBS. We use 200uL per well, each 96 well plate needs 20mL of cells resuspended in media.
\end{description}

\subsection{Cell culture calculations}
\label{sec:cell-culture}
\begin{table}[h]
  \centering
\begin{tabular}{c|c}
Seeding density & 4e4 \\
Wells per plate & 100 \\
Replicates & 4 \\
CONOPS & 2 \\
Number of treatments & 4\\
Number of dilutions & 6\\
Total plates required &  192\\
\end{tabular}

  \end{tabular}
\end{table}

\subsection{Tecan startup}
\label{sec:tecan-start}
\begin{enumerate}
\item Turn Tecan power on. The power switch is located on the white box next to the computer below the deck.
\item Start FluentControl, the Tecan software. This can take 3-5 minutes. Acknowledge the warning that the software is not meant of clinical use.
\item Ensure the deck is cleared of any old tip boxes, tip box lids, plates, stray pipette tips etc.
\item Examine the RGA, the robotic arm fingers
  \begin{enumerate}
  \item Pull down the RGA until it is flat against one of the plate nests. Ensure the fingers are both horizontal. If not, align the fingers.
  \item Firmly but slowly move the fingers up and down to ensure the screw is tight. If the finger is pivoting in the z-direction, tighten the screw at the base of the finger using the TORX15 or TORX20 screw driver.
  \item Spray deck with 70\% ethanol, wipe down.
  \item Place bioharzard bag in the waste chute.
  \end{enumerate}
\end{enumerate}


\subsection{Tecan shutdown}
\label{sec:tecan-shut}
\begin{enumerate}
\item 
\end{enumerate}

\section{Treatments} \label{treatments}
A total of 10 treatments will be prepared and applied to all 60 samples on  \ref{day0instructions}, the treatment conditions have been decided as follows. The following calculations assume we are doing 2 treatment replicates that go into dilutions. We expect 1e6 cells/mL or 1e4 cells/well. We will perform 6x 10-fold serial dilutions to account for carryover.

\begin{table}
  \centering
  \begin{tabular}[]{c|c|c|c}
    \textbf{Organism} & \textbf{Treatments} & \textbf{Treatment} & \textbf{Total plates required} \\
     &  & \textbf{count} & \textbf{(2 CONOPS)} \\
    \hline
    \textit{Bacteria} & Sample  & & \\
    \hline
    & & 1 & 16\\
    \hline
    \textit{Virus} & Sample  & & \\
    \hline
    & & 1 & 16\\
    \hline
    \textit{Fungi} & Sample  & & \\
    \hline
    & & 1 & 16\\
    \hline

  \end{tabular}
  \caption{Summary of dilution plates required for the Day 0 IV\&V run.}
  \label{tab:dilution-plan-day-0}
\end{table}


\begin{table}
  \centering
  \begin{tabular}[]{c|c|c|c}
    \textbf{Organism} & \textbf{Treatments} & \textbf{Treatment} & \textbf{Total plates required} \\
     &  & \textbf{count} & \textbf{(2 CONOPS)} \\
    \hline
    \textit{Bacteria} & PODS 1.5w  & & \\
     & PODS 1.6w  & & \\
     & PODS 2.0  & & \\
     & dry blood  & & \\
     & wet blood  & & \\
    \hline
    & & 5 & 60\\
    \hline
    \textit{Virus} & PODS 1.5v  & & \\
     & PODS 2.0v  & & \\
     & dry blood  & & \\
     & wet blood  & & \\
             & 1:10 wet blood:VTM  & & \\
    \hline
    & & 5 & 60\\
    \hline
    \textit{Fungi} & PODS 1.5  & & \\
     & PODS 2.0  & & \\
     & dry blood  & & \\
     & wet blood  & & \\
    \hline
    & & 4 & 48\\
    \hline

  \end{tabular}
  \caption{Summary of dilution plates required for the Day 7 IV\&V run.}
  \label{tab:dilution-plan-day-7}
\end{table}

\begin{itemize}
    \item Bacteria
        \begin{itemize}
            \item PODS 1.5w
            \item PODS 1.6w
            \item PODS 2.0
            \item PBS
        \end{itemize}
    \item Viruses
        \begin{itemize}
            \item PODS 1.5v
            \item PODS 2.0v
            \item PBS
        \end{itemize}
    \item Fungi
        \begin{itemize}
            \item PODS 1.5
            \item PODS 2.0
            \item PBS
        \end{itemize}
\end{itemize}


\newpage
\section{Preparation for Day 0}\label{sec:day0-prep}
\subsection{Lab Changes}
\begin{enumerate}
    \item \amogh{Moving the centrifuge, and possibly moving Springer lab qPCR next to the Tecan \bradley{is this still do-able?}}
    \item \bradley{We need to open 3-4 full boxes of tips of both sizes}
    \item \zeina{Storage/workplace re-organization}
    \begin{enumerate}
        \item Clear/prepare the plate-filling area, store 384 well plates next to it
        \item Clear and tape off areas for 384 well plate storage after incubation
        \item Clear and tape off the vacuum filtration/treatment preparation area \bradley{I think we should just leave the pipettes i use to do this stuff next to the vacuum filtration/media zone, makes sense that they are there, but also need to clean out the impromptu storage}
    \end{enumerate}
    \item \colin{the declaration and preparation of a fungal zone with the coming sonication bath, a vortex mixer, place to set samples}
\end{enumerate}

\subsection{Labeling System/Plate Maps}
\amogh{create a labeling system for all the media dilution plates, treatment plates, and initial blood plates, or plate maps as needed}

\subsection{Media, plate, and sample preparation}
\textbf{Bacterial media} We need 16 384 well plates filled with 55uL of UBM/well, i.e. 400mL of UBM. Plates need to be labeled and filled by 2025-05-04. 

\textbf{Fungal media} We need 16 384 well plates filled with 55uL of UBM/well, i.e. 400mL of Yeast media (PDB). Plates need to be labeled and filled by 2025-05-04.

\textbf{Seeded Vero plates} We need 16 96 well plates seeded with 200uL of 10\% confluent Vero cells. Plates need to be labeled and filled by 2025-05-04.

\textbf{Drying Caps} We need 30 of the milled drying caps filled with fresh 3a zeolite, and capped with aeraseals, placed into a drying container. 

\textbf{Drying containers} We need all drying container dryrite (if used) 


\section{Preparation for Day 7 and 8}\label{sec:day7-prep}
\textbf{Bacterial media} We need 60 384 well plates filled with 55uL of UBM/well, i.e. 1.5L of UBM. Plates need to be labeled and filled by 2025-05-09.

\textbf{Fungal media} We need 48 384 well plates filled with 55uL of UBM/well, i.e. 1.1L of Yeast media (PDB). Plates need to be labeled and filled by 2025-05-09.

\textbf{Seeded Vero plates} We need 60 96 well plates seeded with 200uL of 10\% confluent Vero cells. Plates need to be labeled and seeded by 2025-05-09 at most. 

\section{Day 0 Order of Operations} \label{day0instructions}

\begin{enumerate}
    \item Check if samples (60x10mLx2 replicates) are actually frozen.
    \item \st{If frozen $\rightarrow$ add half to freezer} Move sample duplicates to freezer to verify by qPCR.
    \item \amogh{This is a precaution against fungi clogging the FCA tips } Spin down tubes briefly.
    \item Convert sample tubes to 1mL 96 well plate format for tecan \amogh{Make Tecan script after optimizing for sample type}
    \item During conversion to 96 well plate format, also pipette a small amount of the sample onto a plate for microscopy (2x 48 wells, or 1x 96well, ~100uL each) \zeina{label microscopy well plates and make plate maps}
    \begin{enumerate}
        \item \bradley{} will observe samples for large fungus clotting, etc.
        \begin{enumerate}
            \item If fungus comes dispersed, maybe handle on robot already 
            \item If large fungal particles are observed, hand 5 mL of each sample to the Colin \colin{}, and he can handle them with sonication and mixing (per his protocol) before handing back to Tecan
        \end{enumerate}
    \end{enumerate}
    \item At this point, the samples need to be handled on the robot for the following 3 tasks in order:
    \begin{enumerate}
        \item Day 0 identification (what is in each sample, if Cell Viability Core remains blinded)
        \item dilution into treatment plates, and set up to dry
        \item Day 0 viability (how many are in each sample) 
    \end{enumerate}
    
    \item All 60 samples will be treated separately according to \ref{treatments}, summarized below:
    
    \begin{enumerate}
        \item 4 treatments which will be handed as “bacterial” samples
    \end{enumerate}
    \begin{enumerate}
        \item 3 treatments which will be handled as “viral” samples
    \end{enumerate}
    \begin{enumerate}
        \item 3 treatments which will be handled as “fungal” samples
    \end{enumerate}
\end{enumerate}

\section*{Day 1 Order of Operations} \label{day1instructions}


\section*{Day 7 Order of Operations} \label{day7instructions}

\section*{Day 8 Order of Operations} \label{day7instructions}


\end{document}

%%% Local Variables:
%%% mode: LaTeX
%%% TeX-master: t
%%% End:
