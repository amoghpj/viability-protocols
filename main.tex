\documentclass{article}

\usepackage{graphicx}
\usepackage{amsmath}
\usepackage{soul}
\usepackage{hyperref}
\usepackage{cleveref}
\usepackage{xcolor}
\usepackage[]{fancyhdr}
\usepackage[margin=1in]{geometry}

\title{Protocol and tasks - MUPPETS viability}
\newcounter{comments}
\newcommand{\amogh}[1]{{\addtocounter{comments}{1}}{\color{red}{[\textbf{TODO Amogh \thecomments :}\ #1}]}}
\newcommand{\bradley}[1]{{\addtocounter{comments}{1}}{\color{violet}{[\textbf{TODO Bradley \thecomments :}\ #1]}}}
\newcommand{\zeina}[1]{{\addtocounter{comments}{1}}{\color{blue}{[\textbf{TODO Zeina \thecomments :}\ #1]}}}
\newcommand{\colin}[1]{{\addtocounter{comments}{1}}{\color{cyan}{[\textbf{TODO Colin \thecomments :}\ #1]}}}
\date{\today}


\begin{document}
\pagestyle{fancy}
\fancyfoot{}
\fancyhead{}
\fancyfoot[LO,RE]{\thepage \quad   LAST UPDATED \today}

\maketitle
\tableofcontents


\section{Task List}



\newpage

\section{Foreword}
We are receiving 60 distinct samples, with 2 replicates from NIST. These samples are 10 mL blood which are potentially frozen, and potentially not frozen. Each sample has between 0 and 3 organisms, which are a combination of fungi, viruses, and bacteria. We need to store the sample in 2 CONOPS (doubling treatment plates). The Cell Viability Core (CVC) are potentially unblinded to the samples. The treatment teams (Silver \& Marelli) are blinded to the samples to avoid preferential treatment. We need to report for each sample what was in it, how much, and how much survived after 7 days.

All samples need to be treated as BSL2+. All well plates need to have lids at all times. All handling of samples need to be done in the Tecan (sealed) or in a bio-safety cabinet. All biosafety bags which handle BSL2+ samples need to be sprayed down before placing in normal biosafety containers, including that from the Tecan. 

\section{Protocols}
\begin{itemize}
    \item Media prep (\Cref{sec:media-prep})
    \item Cell culture calculations
    \item Tecan startup (\Cref{sec:tecan-start})
    \item Tecan shutdown (\Cref{sec:tecan-shut})
    \item Protocol - Bacterial, fungal serial dilutions (\Cref{sec:serial-dil-bacterial})
    \item Protocol - Viral serial dilutions (\Cref{sec:serial-dil-viral})
    \item Protocol - Sample identification with qPCR  (\Cref{sec:sample-ident})
    \item Protocol - qPCR setup for bacteria/fungi 
    \item Protocol - qPCR setup for virus 
    \item Protocol - OD reads (Method: \texttt{384 well plates read})
    \item Treatment Preparation and Conditions 
    \item Handling of samples with hyphal fungi 
    \begin{itemize}
        \item receive 60 samples (4 ml of blood in ORIGINAL 10 ML TUBES) handed to him by Amogh/Bradley
        \item Vortex ORIGINAL tubes, 1-2 at a time, for 4 rounds, 34 seconds each. (The breaks in between rounds are for Colin's poor fingers to rest). \textbf{[10 mins]}
        \item Transfer vortexed tubes to 5 tube racks (12 tubes per rack).
        \item Place first rack in bath sonicator. Make sure water is up to "fill" level.
        \item Turn on bath sonicator for 10 minutes. 
        \begin{itemize}
            \item 
        \end{itemize}
        \item Repeat for all 5 racks. \textbf{[1 h 15 mins]}
        \begin{itemize}
            \item Look for breakage of fungus ball clumps. If you're not seeing anything, possibly repeat certain samples.
        \end{itemize}
        \item Grab pre-prepared \zeina{plan to hand colin these labeled tubes} 5 ml tubes, clean, labeled 1-60, and put 100 um cell strainer on top
        \begin{itemize}
            \item  (make sure the width of the cell strainer is compatible with the 5 ml tubes not the 50 mls!!)
        \end{itemize}
    \item Pour or use serological pipette \colin{decide which one} to filter the 60 sonicated samples into the clean 5 ml labeled tubes. 
    \item For any remaining very large fungal balls stuck in filter, we would like to save them just in case. That way we are able to qPCR to identify the strain. \zeina{prepare a rack of 50 ml conicals on standby (ulabeled) just in case}
    \end{itemize}
\end{itemize}

\subsection{Media prep}
\label{sec:media-prep}
\begin{description}
\item[Universal bacterial media (UBM)] TSB+BHI supplemented with NAD and Hemin. TSB - Mix 30g TSB powder into 1L media, autoclave. BHI - mix 37g BHI powder into 1L media, autoclave. 2000X Hemin and NAD stocks are stored in the -80 in GOLD513.
\item[Fungal media] Potato Dextrose Broth + PenStrep.
\item[Cell culture media for Vero cells] EMEM + 10\% FBS. We use 200uL per well, each 96 well plate needs 20mL of cells resuspended in media.
\end{description}

\subsection{Cell culture calculations}
\label{sec:cell-culture}
\begin{table}[h]
  \centering
\begin{tabular}{c|c}
Seeding density & 4e4 \\
Wells per plate & 100 \\
Replicates & 4 \\
CONOPS & 2 \\
Number of treatments & 4\\
Number of dilutions & 6\\
Total plates required &  192\\
\end{tabular}

\end{table}

\subsection{Tecan startup}
\label{sec:tecan-start}
\begin{enumerate}
\item Turn Tecan power on. The power switch is located on the white box next to the computer below the deck.
\item Start FluentControl, the Tecan software. This can take 3-5 minutes. Acknowledge the warning that the software is not meant of clinical use.
\item Ensure the deck is cleared of any old tip boxes, tip box lids, plates, stray pipette tips etc.
\item Initialize the instrument. On the menu bar on top, Click \texttt{Run > Initialize instrument}.
\item Examine the RGA, the robotic arm fingers
  \begin{enumerate}
  \item Pull down the RGA until it is flat against one of the plate nests. Ensure the fingers are both horizontal. If not, align the fingers.
  \item Firmly but slowly move the fingers up and down to ensure the screw is tight. If the finger is pivoting in the z-direction, tighten the screw at the base of the finger using the TORX20 screw driver (stored in drawer directly below QuantStudio).
  \item Spray deck with 70\% ethanol, wipe down.
  \item Place bioharzard bag in the waste chute.
  \end{enumerate}
\end{enumerate}


\subsection{Tecan shutdown}
\label{sec:tecan-shut}
\begin{enumerate}
\item Take the biohazard bag out of the trash chute carefully.
\item Clear any remaining labware.
\item Spray down with 70\% ethanol. Let sit for 5 minutes, wipe down thoroughly.
\item CLose the FluentControl software.
\item Power off the Tecan using the switch on the white controller box placed below the deck.
\end{enumerate}

\subsection{Bacterial/Fungal Serial dilutions}
\label{sec:serial-dil-bacterial}
NIST samples are expected to have 1e6 cells/mL. We are assaying viability of ~10$\mu$L of sample, or about 1e4 cells. We will perform 6 serial dilutions per treatment.

\begin{itemize}
\item Pre-fill 384 well clear plates with UBM (\Cref{sec:media-prep}) for each day.
\item  Load serial dilution plates into the Cytomat. 
\item Fill a 100mL trough with sterile water.
\item Follow further instructions  provided by method: \texttt{Serial Dilutions - Bacterial/Fungal}).
\end{itemize}

\subsection{Viral dilutions}
We will need 4 seeded plates per dilution. If unblinded, we will need 24 seeded 96 well plates for 6 dilutions.
\label{sec:serial-dil-viral}
\begin{itemize}
\item Follow instructions on method: \texttt{Serial Dilutions - Viral})
\end{itemize}

\subsection{Sample identification with qPCR }
\label{sec:sample-ident}
\textit{A priori} the sample composition is unknown, but we have designed probes for the 41 species that we expect to comprise the samples.



On Day 0 we will assay all 60 samples with all probes. 


\begin{itemize}
\item Prepare 4 96 well skirted plates. Each column has 1 MM.
\item Stamp out 384 well qPCR plates with 6 probe mixes at a time, stamp the sample plate onto the qPCR plate.
\item Assay 6 targets at a time.
\end{itemize}

\section{Treatments} \label{treatments}
A total of 10 treatments will be prepared and applied to all 60 samples on  \ref{day0instructions}, the treatment conditions have been decided as follows. The following calculations assume we are doing 2 treatment replicates that go into dilutions. We expect 1e6 cells/mL or 1e4 cells/well. We will perform 6x 10-fold serial dilutions to account for carryover. See \ref{tab:dilution-plan-day-0} for a reference of the treatments and how many plates, and dilution plates are required. For the preparation of solutions, double the treatment per each organism type, and each treatment type. For treatment plates, each plate will have 36 microliters of treatment, to which 14 microliters of sample will be placed, or a 1/3 dilution. This has been verified for blood work.

\begin{table} 
  \centering
  \begin{tabular}[]{c|c|c|c}
    \textbf{Organism} & \textbf{Treatments} & \textbf{Treatment} & \textbf{Total plates required} \\
     &  & \textbf{count} & \textbf{(2 CONOPS)} \\
    \hline
    \textit{Bacteria} & Sample  & & \\
    \hline
    & & 1 & 16\\
    \hline
    \textit{Virus} & Sample  & & \\
    \hline
    & & 1 & 16\\
    \hline
    \textit{Fungi} & Sample  & & \\
    \hline
    & & 1 & 16\\
    \hline

  \end{tabular}
  \caption{Summary of dilution plates required for the Day 0 IV\&V run.}
  \label{tab:dilution-plan-day-0}
\end{table}


\begin{table}
  \centering
  \begin{tabular}[]{c|c|c|c|c}
    \textbf{Organism} & \textbf{Treatments}  & \textbf{Treatment} & \textbf{Total plates required} & \textbf{Plate ID}\\
     &   & \textbf{count} & \textbf{(2 CONOPS)} & \\
    \hline
    \textit{Bacteria} & PODS 1.5w   & & & 1\\
     & PODS 1.6w   & & & 2\\
     & pullulan & & & 3\\
     & dry blood   & & & 4\\
     & wet blood   & & & 5\\
    \hline
    &  & 5& 60& \\
    \hline
    \textit{Virus} & PODS 1.5w & & & 6\\
     & PODS 1.5WV & & & 7\\
     & dry blood   & & & 8\\
     & wet blood   & & & 9\\
             & 1:10 wet blood:VTM   & & & 10\\
 & pullulan & & & 11\\
    \hline
    &  & 6& 60 & \\
    \hline
    \textit{Fungi} & PODS 1.5 & & & 12\\
     & PODS 1.6 & & & 13\\
     & dry blood   & & & 14\\
     & wet blood   & & & 15\\
    \hline
    &  & 4 & 48 & \\
    \hline

  \end{tabular}
  \caption{Summary of dilution plates required for the Day 7 IV\&V run.}
  \label{tab:dilution-plan-day-7}
\end{table}

\newpage
\section{Preparation for Day 0}\label{sec:day0-prep}
\subsection{Lab Changes}
\begin{enumerate}
    \item \bradley{We need to open 6 full boxes of tips of both sizes}
    \item \zeina{Storage/workplace re-organization}
    \begin{enumerate}
        \item Clear/prepare the plate-filling area, store 384 well plates next to it
        \item Clear and tape off areas for 384 well plate storage after incubation
        \item Clear and tape off the vacuum filtration/treatment preparation area \bradley{I think we should just leave the pipettes i use to do this stuff next to the vacuum filtration/media zone, makes sense that they are there, but also need to clean out the impromptu storage}
    \end{enumerate}
    \item \colin{the declaration and preparation of a fungal zone with the coming sonication bath, a vortex mixer, place to set samples}
\end{enumerate}

\subsection{Labeling System/Plate Maps}
\label{sec:labeling}

The IV\&V experiment ID is \textsc{MP52}.
\subsubsection{Sample handling}
Template: \textsc{MP52}-\texttt{S<SAMPLE ID>}. See \Cref{tab:sample-map} for layout.

Template for consolidated samples: \texttt{MP52}-\texttt{BACTERIA}, \texttt{MP52}-\texttt{FUNGI}, \texttt{MP52}-\texttt{VIRUS}. 

See \Cref{tab:sample-map} for layout. 

\begin{table}
\centering
\begin{tabular}{l|rrrrrrrrrrrr}
 & 1 & 2 & 3 & 4 & 5 & 6 & 7 & 8 & 9 & 10 & 11 & 12\\
\hline
A & 1 & 9 & 17 & 25 & 33 & 41 & 49 & 57 &  &  &  & \\
B & 2 & 10 & 18 & 26 & 34 & 42 & 50 & 58 &  &  &  & \\
C & 3 & 11 & 19 & 27 & 35 & 43 & 51 & 59 &  &  &  & \\
D & 4 & 12 & 20 & 28 & 36 & 44 & 52 & 60 &  &  &  & \\
E & 5 & 13 & 21 & 29 & 37 & 45 & 53 &  &  &  &  & \\
F & 6 & 14 & 22 & 30 & 38 & 46 & 54 &  &  &  &  & \\
G & 7 & 15 & 23 & 31 & 39 & 47 & 55 &  &  &  &  & \\
H & 8 & 16 & 24 & 32 & 40 & 48 & 56 &  &  &  &  & \\
\end{tabular}
\caption{Sample ID map.}
\label{tab:sample-map}
\end{table}

\subsubsection{Treatment labeling}
Template: \textsc{MP52}-\texttt{<PLATEID>}. See \Cref{tab:dilution-plan-day-7} for metadata.

\subsubsection{Dilution labeling}

Template: \textsc{MP52}-\texttt{<PLATEID>}-\texttt{<DILUTIONID>}. 
We will perform 6 dilutions per sample.

\subsection{Media, plate, and sample preparation}
\textbf{Bacterial media} We need 16 384 well plates filled with 55uL of UBM/well, i.e. 400mL of UBM. Plates need to be labeled and filled by 2025-05-04. 

\textbf{Fungal media} We need 16 384 well plates filled with 55uL of UBM/well, i.e. 400mL of Yeast media (PDB). Plates need to be labeled and filled by 2025-05-04.

\textbf{Seeded Vero plates} We need 16 96 well plates seeded with 200uL of 10\% confluent Vero cells. Plates need to be labeled and filled by 2025-05-04.

\textbf{Drying Caps} We need 30 of the milled drying caps filled with fresh 3a zeolite, and capped with aeraseals, placed into a drying container. 

\textbf{Drying containers} We need all drying containers filled with zeolite, dryrite (if used) 

\textbf{Tubes for Dividing Samples Upon Receipt} We need to pre-label and designate the tubes into three sets: Amogh, C

\section{Preparation for Day 7 and 8}\label{sec:day7-prep}
\textbf{Bacterial media} We need 60 384 well plates filled with 55uL of UBM/well, i.e. 1.5L of UBM. Plates need to be labeled and filled by 2025-05-14.

\textbf{Fungal media} We need 48 384 well plates filled with 55uL of UBM/well, i.e. 1.1L of Yeast media (PDB). Plates need to be labeled and filled by 2025-05-14.

\textbf{Seeded Vero plates} We need 60 96 well plates seeded with 200uL of 10\% confluent Vero cells. Plates need to be labeled and seeded by 2025-05-13 at most. 

\section{Day 0 Order of Operations} \label{day0instructions}
\textbf{NOTE:} Leave all sample tubes on ice when not actively being handled. Dry tubes before placing on Tecan deck.
\begin{enumerate}
    \item GOTO Tecan, and run the startup verification script to make sure that it is lifting plates, grabbing pipettes tips, and that the cytomat is working. 
    \item Check if samples (60x10mLx2 replicates) are actually frozen. 
    \item Move sample duplicates to freezer to verify by qPCR.
    \item For \ul{bacteria}: Consolidate sample tubes and transfer 0.5mL to 2x 0.8mL 96 well deep well plates (\texttt{MP52-BACTERIA}). \textit{9mL remaining}. Stamp 50uL onto 96 well flat bottomed plate.  Bradley will optionally do microscopy on plates to look for fungi.  \textbf{\textsc{15 minutes}}
    
    \item For \ul{virus}: Centrifuge 60 tubes 16 at a time at 7000g for 5 minutes. Consolidate by transferring 0.5mL onto 0.8mL 96 well deep well plate (\texttt{MP52-VIRUS}). \textit{8.5mL remaining}. \textbf{\textsc{1 hour}}
    \item For \ul{fungi}: Hand spun down tubes to Colin, who will vortex tubes, and follow fungal sonication protocol . Consolidate onto third prechilled 0.8mL deep well plate (\texttt{MP52-FUNGI}). \textit{8mL remaining}. \textbf{\textsc{1.5 hours}}
    \item At this point, we will have 3 0.8mL deep well plates enriched for each organism type. Approximately 8mL of samples remain in original sample tubes. Hand off tubes to 
    \item Next tasks:
    \begin{enumerate}
    \item Setup formulations in round bottom treatment plates.
        \item Day 0 identification - 7 qPCR plates. See \Cref{sec:sample-ident}.
        \item Stamp samples to treatment plates.
        \item Serial dilutions into bacterial media, fungal media (\textbf{\textsc{30 minutes each}}), and Vero plates (\textbf{\textsc{45 minutes}}).
    \end{enumerate}
    
    \item All 60 samples will be treated separately according to \Cref{treatments}, summarized below:
    
    \begin{enumerate}
        \item 4 treatments which will be handed as “bacterial” samples
    \end{enumerate}
    \begin{enumerate}
        \item 3 treatments which will be handled as “viral” samples
    \end{enumerate}
    \begin{enumerate}
        \item 3 treatments which will be handled as “fungal” samples
    \end{enumerate}
\end{enumerate}

\section*{Day 1 Order of Operations} \label{day1instructions}
\begin{enumerate}
    \item On Tecan: run the startup verification script to make sure that it is lifting plates, grabbing pipettes tips, and that the cytomat is working. 
    \item Take out the incubating plates from day 0 viability to the benchtop to prevent evaporation
    \item Prepare the dilution plates for day 1 viability of all treatment plates
\end{enumerate}

\section*{Day 2 Order of Operations} \label{day1instructions}
\begin{enumerate}
    \item Take out the incubating plates from day 1 viability to the benchtop to prevent evaporation
    \item Do the qPCR viability protocol for bacteria \textbf{day 0} plates to measure viability (round \textbf{1})
    \item Do the qPCR viability protocol for fungal \textbf{day 0} plates to measure viability (round \textbf{1})
\end{enumerate}

\section*{Day 3 Order of Operations} \label{day1instructions}
\begin{enumerate}
    \item Do the qPCR viability protocol for bacteria \textbf{day 0} plates to measure viability (round \textbf{2})
    \item Do the qPCR viability protocol for fungal \textbf{day 0} plates to measure viability (round \textbf{2})
\end{enumerate}

\section*{Day 7 Order of Operations MAY 15 2025} \label{day7instructions}
\begin{enumerate}
    \item GOTO Tecan, and run the startup verification script to make sure that it is lifting plates, grabbing pipettes tips, and that the cytomat is working. 
    \item Prepare the dilution plates for viability for the frozen samples
    \item Place the RT bacterial plates into the 37C incubator
    \item Place the RT fungal plates on the bench
    \item place the RT viral plates into the mammalian incubator
\end{enumerate}

\section*{Day 8 Order of Operations MAY 16 2025} \label{day8instructions}
\begin{enumerate}
    \item GOTO Tecan, and run the startup verification script to make sure that it is lifting plates, grabbing pipettes tips, and that the cytomat is working. 
    \item Take out the incubating plates from day 7 viability to the benchtop to prevent evaporation
    \item Prepare the dilution plates for viability for the frozen samples
    \item Place the RT bacterial plates onto the benchtop
    \item place the FR bacterial plates into the 37 incubator
    \item place the FR fungal plates on the bench
    \item place the FR viral plates into the mammalian incubator
\end{enumerate}

\section*{Day 9 Order of Operations MAY 17 2025} \label{day8instructions}
\begin{enumerate}
    \item GOTO Tecan, and run the startup verification script to make sure that it is lifting plates, grabbing pipettes tips, and that the cytomat is working.
    \item PLACE RT bacterial plates into the -80
    \item Take the FR bacterial plates out of the incubator
    %\item Take out the incubating plates from day 8 viability to the benchtop to prevent evaporation
    %\item Do the qPCR viability protocol for bacteria \textbf{day 7 RT} plates to measure viability (round \textbf{1})
    %\item Do the qPCR viability protocol for fungal \textbf{day 7 RT} plates to measure viability (round \textbf{1})
\end{enumerate}

\section*{Day 10 Order of Operations MAY 18 2025} \label{day8instructions}
\begin{enumerate}
    \item GOTO Tecan, and run the startup verification script to make sure that it is lifting plates, grabbing pipettes tips, and that the cytomat is working. 
    \item PLACE the FR bacterial plates into the -80
    %\item Do the qPCR viability protocol for bacteria \textbf{day 7 frozen} plates to measure viability (round \textbf{2})
    %\item Do the qPCR viability protocol for fungal \textbf{day 7 frozen} plates to measure viability (round \textbf{2})
\end{enumerate}

\section*{Day 12 Order of Operations MAY 20 2025} \label{day8instructions}
\begin{enumerate}
    \item GOTO Tecan, and run the startup verification script to make sure that it is lifting plates, grabbing pipettes tips, and that the cytomat is working. 
    \item PLACE the RT fungal plates into the -80
    \item PLACE the RT viruses into the -80
    %\item Do the qPCR viability protocol for bacteria \textbf{day 7 frozen} plates to measure viability (round \textbf{2})
    %\item Do the qPCR viability protocol for fungal \textbf{day 7 frozen} plates to measure viability (round \textbf{2})
\end{enumerate}

\section*{Day 13 Order of Operations MAY 21 2025} \label{day8instructions}
\begin{enumerate}
    \item GOTO Tecan, and run the startup verification script to make sure that it is lifting plates, grabbing pipettes tips, and that the cytomat is working. 
    \item PLACE the FR fungal plates into the -80
    \item PLACE the FR viruses into the -80
    %\item Do the qPCR viability protocol for bacteria \textbf{day 7 frozen} plates to measure viability (round \textbf{2})
    %\item Do the qPCR viability protocol for fungal \textbf{day 7 frozen} plates to measure viability (round \textbf{2})
\end{enumerate}

\section*{Appendix - Plate Maps} \label{platemaps}



\end{document}

%%% Local Variables:
%%% mode: LaTeX
%%% TeX-master: t
%%% End:
